
% PREAMBLE ===============================================================

\documentclass[a4paper,11pt]{article}
\usepackage[german,ngerman]{babel}
\usepackage[utf8]{inputenc}
\usepackage[T1]{fontenc}
\usepackage[top=1.3in, bottom=1in, left=1.0in, right=0.6in]{geometry}
\usepackage{lmodern}
\usepackage{amssymb}
\usepackage{mathtools}
\usepackage{amsmath}
\usepackage{enumerate}
\usepackage{pgfplots}
\usepackage{breqn}
\usepackage{tikz}
\usepackage{fancyhdr}
\usepackage{multicol}

\usetikzlibrary{calc}
\usetikzlibrary{patterns}
\usetikzlibrary{decorations.markings,arrows,automata}
\usetikzlibrary{positioning}

\newcommand{\authorinfo}{Katja Möhring, Stephan Tietz, Carolin Konietzny}
\newcommand{\titleinfo}{GWV, Blatt 08}

\author{\authorinfo}
\title{\titleinfo}
\date{\today}

\pagestyle{fancy}
\fancyhf{}
\fancyhead[L]{\authorinfo}
\fancyhead[R]{\titleinfo}
\fancyfoot[C]{\thepage}

\begin{document}
\maketitle
\begin{enumerate}
\item[\textbf{1.3}]
    Da die Graphik keine gerichteten Kanten hat, kann man daraus nicht genau ablesen, wie die Abhängikeiten zwischen den Komponenten sind.
   
    Wir lassen also Allgemeinwissen über die Domäne in unsere Betrachtung einfließen und gehen von folgendem Belief Network aus:


    \begin{tikzpicture}[->,>=stealth',shorten >=1pt,thick,scale=0.6,node distance=3cm, every node/.style={scale=0.6}]
        \node[minimum width=2cm,draw,circle] (b) {Battery};
        \node[minimum width=2cm,draw,circle] (i) [below left=of b] {Ignition};
        \node[minimum width=2cm,draw,circle] (efr) [below right=of b] {ElFuelReg};
        \node[minimum width=2cm,draw,circle] (s) [below of=i] {Starter};
        \node[minimum width=2cm,draw,circle] (fp) [below of=efr] {FuelPump};
        \node[minimum width=2cm,draw,circle] (f) [below of=fp] {Filter};
        \node[minimum width=2cm,draw,circle] (e) [below left=of f] {Engine};
        \node[minimum width=2cm,draw,circle] (ft) [right of=efr] {FuelTank};

        \path (b) edge (i);
        \path (b) edge (efr);
        \path (i) edge (s);
        \path (s) edge (e);
        \path (i) edge (efr);
        \path (efr) edge (fp);
        \path (fp) edge (f);
        \path (f) edge (e);
        \path (ft) edge (fp);
    \end{tikzpicture} 


    Wir nehmen an, dass $p(x)$ die Wahrscheinlichkeit ist, dass $x$ funktioniert, ohne andere Abhängigkeiten zu beachten (in diesem Fall gilt $\forall x: p(x) = 0.9$).
    \begin{align*}
    P(Battery) &= p(Battery) = 0.9 \\
    P(Starter) &= p(Battery) * p(Ignition) * p(Starter) = 0.9^3 \\
    P(Engine)  &= \prod_{x \in Komponenten} p(x) = 0.9^{|Komponenten|} = 0.9^8
    \end{align*}

    Wenn man davon ausgeht, dass die Fuel Pump funktioniert, kann man daraus schließen, dass auch Battery, Ignition, EFR und Fuel Tank intakt sind. Es bleiben also noch Starter, Filter und Engine, welche defekt sein können.

    $P'(Engine) = p(Starter) * p(Filter) * p(Engine) = 0.9^3 $
\end{enumerate}
\end{document}